\documentclass[40pt]{article}

% Encoding.
\usepackage{geometry}
\usepackage[T2A]{fontenc}
\usepackage[utf8]{inputenc}
\usepackage[english,russian]{babel}

% Code insertion.
\usepackage[outputdir=temp]{minted}

% Math functions.
\usepackage{amsmath}

% Image insertion.
\usepackage{svg}

% No line breaks.
\usepackage[none]{hyphenat}

% ToC hyperlinks.
\usepackage{hyperref}
\hypersetup{
	colorlinks,
	citecolor=black,
	filecolor=black,
	linkcolor=black,
	urlcolor=black
}

\title{Анализ алгоритмов}
% \author{Смирнов Александр}
\date{\today}

\begin{document}

\maketitle

% \newpage
\tableofcontents
\newpage







\section{Скорость роста длины записи коэффициентов при реализации метода Гаусса (10)}

$$\left\|b_{i j}^{r}\right\| \leq(r+1) M+r^{2}$$



\begin{itemize}
	\item $||a||$ -- длина записи числа $a$
	\item $M$ -- максимальная длина записи элементов матрицы
	\item $r$ -- ранг матрицы
	\item $||b_{ij}^r||$ -- длина записи коэффицента после $r$-й итерации
\end{itemize}


\section{Представление ''длинного'' числа в файле (массиве, списке) как числа в системе счисления по модулю $p$ ($p=1000$, если \texttt{integer} $2^{16}$, если $p=10 000 000$ \texttt{longinteger} $2^{32}$). Запись из файла. Оценка числа шагов. Вывод в файл. Оценка числа шагов (14)}


\subsection{Представление длинного числа}

Всякое целое неотрицательное число $x$ может быть представлено в $m$-ичной системе счисления (при $m \geq 2)$ в виде $x=m^{k-1} x_{0}+m^{k-2} x_{1}+\ldots m x_{k-2}+x_{k-1}.$ При этом $k$ -- длина записи $m$-ичного представления числа $x, 0 \leq x_{i} \leq m-1$ при $i=0, \ldots, k-1$

\subsection{Запись из файла}

\textbf{Оценка числа шагов:} квадратичное от длины записи исходного числа в файле количество ''шагов''.

Пусть в файле записано десятичное число, заданное словом $a_{1} \cdots a_{n}\left(0 \leq a_{i} \leq 9\right)$. Требуется представить его динамическим массивом (или списком).

Под ''шагом'' понимается одна из следуюших операций:
считывание цифры из файла, запись цифры в целочисленный массив, выделение первой цифры многозначного числа и её удаление из него, приписывание цифры в конец числа. Заметим, что эти ''шаги'' не равнозначны, т.к. последние два требуют нахождения остатка от деления на $10$, а также умножения на 10 и сложения.

\subsection{Вывод в файл}

\textbf{Оценка числа шагов:} линейное от длины записи исходного числа количество ''шагов''.

При выводе числа необходимо помнить, что в каждом элементе массива, в котором хранится многоразрядное число, записана не последовательность цифр, а число, записанное этими цифрами. Поэтому число, десятичная запись которого меньше, чем длина записи выбранного нами основания $m$, необходимо дополнить ведущими нулями.

Под ''шагом'' будем понимать одну из следующих операций: запись ''макроцифры'' в символьную переменную, сравнение длины записи ''макроцифры'' с $\|m-1\|$, дополнение строки ведущим нулём.



\section{Сложение двух ''длинных'' положительных чисел. Оценка числа шагов (17)}

\textbf{Оценка числа шагов:} общее число ''шагов'' при сложении двух неотрицательных чисел не превосходит $3\max \{A[0], B[0]\}+1$, то есть составляет $\mathrm{O}(\max \{A[0], B[0]\})$

Чтобы сложить два неотрицательных многоразрядных числа, записанных в массивы $A$ и $B$, достаточно последовательно складывать по модулю $m$ числа, записанные в $A[i], B[i]$ и $d[i]$ для $i=$ $1, \ldots, \max \{A[0], B[0]\}$, где $d[1]=0$, при $i>1, d[i]-$ это $1$ (если $A[i-1]+B[i-1]+d[i-1]>m$) или 0 в противном случае.

При подсчёте числа шагов в этом разделе под ''шагом'' понимается одна из следующих операций: вычисление $A[i-1]+B[i-1]+d[i-1]$ $\bmod m$, проверка условия $A[i-1]+B[i-1]+d[i-1]>m$ и вычисление $d[i]$


\section{Предикаты равенства и неравенств ''длинных'' положительных чисел. Оценка числа шагов (18)}

\textbf{Оценка числа шагов:} если под ''шагом'' понимать количество сравнений ''макроцифр'', то обшее число ''шагов'' такой процедуры не превосходит $A[0]$. В общем случае число ''шагов'' вычисления каждого из четырёх предикатов не превосходит $\min \{A[0], B[0]\}$.


Оценим число шагов вычисления значений предикатов $x=y$ и $x<y$ для случая, когда $A[0]=B[0]$.

Начиная со старшего разряда (то есть с $A[A[0]]$ и $B[B[0]])$ сравниваем значения чисел в $A[i]$ и $B[i]$ до тех пор, пока они совпадают. Если для некоторого $i_{0} A\left[i_{0}\right] \neq B\left[i_{0}\right],$ то $x \neq y .$ Если при этом $A\left[i_{0}\right]<B\left[i_{0}\right],$ то $x<y,$ если $A\left[i_{0}\right]>B\left[i_{0}\right],$ то $x>y$


\section{Вычитание двух ''длинных'' положительных чисел. Оценка числа шагов (18)}

\textbf{Оценка числа шагов:} общее число  ''шагов'' при вычитании двух положительных чисел не превосходит $4 \max \{A[0], B[0]\}+1$, то есть составляет $\mathrm{O}(\max \{A[0], B[0]\})$

При подсчёте числа шагов в этом разделе под ''шагом'' понимается одна из следуюшци операций: вычисление $A[i-1]-B[i-1]-d[i-1]$ $(\bmod m)$, проверка условия $A[i-1]+B[i-1]+d[i-1]>0$ и вычисление $d[i]$. Кроме того, предварительно проверяется условие $x \geq y$.




\section{Умножение ''длинного'' числа на короткое. Оценка числа шагов}
\section{Умножение ''длинных'' чисел. Оценка числа шагов}
\section{Деление ''длинных'' чисел. Оценка числа шагов}
\section{Оценки числа шагов метода Гауса при действиях с ''длинными'' числами}
\section{Сортировки и оценки числа их шагов: Пузырёк. Сортировка вставками. Сортировка слияниями фон Неймана}
\section{Алгоритмы на графах, различные способы представления графа в компьютере}
\section{Алгоритм поиска в глубину. Оценки числа шагов в зависимости от способа представления графа}
\section{Алгоритм поиска в ширину. Оценки числа шагов в зависимости от способа представления графа}
\section{Задачи, решаемые с помощью этих алгоритмов:— выделение компонент связности,— проверка на двудольность и выделение долей,— выделение остова графа}
\section{Нахождение остова минимального веса. Метод Р. Прима. Оценки числа шагов}
\section{Алгоритм Дейкстры поиска кратчайшего пути. Оценки числа шагов}
\section{Нахождение циклов и мостов в графе.  Оценки числа шагов}
\section{Эйлеров цикл.  Оценки числа шагов}
\section{Гамильтонов цикл.  Оценки числа шагов}
\section{Алгоритм генерации всех независимых множеств.  Оценки числа шагов}
\section{Теорема о НМ, ВП, КЛИКА.  Оценки числа шагов}
\section{Отличия между интуитивным и математическим понятиями}
\section{Машины Тьюринга и их модификации. Тезис Тьюринга-Чёрча}
\section{Теорема о числе шагов МТ, моделирующей работу k-ленточной МТ}
\section{Недетерминированные МТ.  Теорема о числе шагов МТ, моделирующей работу недетерминированной МТ}
\section{Понятия сложности алгоритма от данных, сложность алгоритма, сложность задачи. Верхняя и нижняя оценки сложности}
\section{Соотношение между временем работы алгоритма требуемой памятью}
\section{Классы алгоритмов и задач. Схема обозначений}
\section{Классы $P$, $NP$ и $P-SPACE$. Соотношения между этими классами}
\section{Полиномиальная сводимость и полиномиальная эквивалентность}
\section{Полиномиальная сводимость задачи ГЦ к задаче КОМИВОЯЖЁР}
\section{Классы эквивалентности по отношению полиномиальной эквивалентности. Класс P – пример такого класса}
\section{NP-полные задачи. Класс NP-полных задач — класс эквивалентности по отношению полиномиальной эквивалентности}
\section{Задача ВЫПОЛНИМОСТЬ (ВЫП).Теорема Кука}
\section{Задача 3-ВЫПОЛНИМОСТЬ (3-ВЫП). Её NP-полнота}
\section{Задачи ВЕРШИННОЕ ПОКРЫТИЕ (ВП), НЕЗАВИСИМОЕ МНОЖЕСТВО (НМ), КЛИКА.  NP-полнота задачи ВП.  Полиномиальная эквивалентность этих трёх задач}
\section{NP-полнота задач ГЦ и ГП (без доказательства)}
\section{NP-полнота задач 3-С и РАЗБИЕНИЕ (без доказательства)}
\section{Метод сужения доказательства NP-полноты}
\section{''Похожие'' задачи и их сложность}
\section{Анализ подзадач}
\section{Алгоритм решения задачи РАЗБИЕНИЕ}
\section{Задачи с числовыми параметрами. Псевдополиномиальные задачи}

\end{document}
