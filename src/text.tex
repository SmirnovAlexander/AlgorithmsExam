\documentclass[40pt]{article}

% Encoding.
\usepackage{geometry}
\usepackage[T2A]{fontenc}
\usepackage[utf8]{inputenc}
\usepackage[english,russian]{babel}

% Code insertion.
\usepackage[outputdir=temp]{minted}

% Math functions.
\usepackage{amsmath}

% Image insertion.
\usepackage{svg}

% No line breaks.
\usepackage[none]{hyphenat}

% ToC hyperlinks.
\usepackage{hyperref}
\hypersetup{
	colorlinks,
	citecolor=black,
	filecolor=black,
	linkcolor=black,
	urlcolor=black
}

\title{Анализ алгоритмов}
% \author{Смирнов Александр}
\date{\today}

\begin{document}

\maketitle

% \newpage
\tableofcontents
\newpage







\section{Скорость роста длины записи коэффициентов при реализации метода Гаусса (10)}

$$\left\|b_{i j}^{r}\right\| \leq(r+1) M+r^{2}$$



\begin{itemize}
	\item $||a||$ -- длина записи числа $a$
	\item $M$ -- максимальная длина записи элементов матрицы
	\item $r$ -- ранг матрицы
	\item $||b_{ij}^r||$ -- длина записи коэффицента после $r$-й итерации
\end{itemize}


\section{Представление ''длинного'' числа в файле (массиве, списке) как числа в системе счисления по модулю $p$ ($p=1000$, если \texttt{integer} $2^{16}$, если $p=10 000 000$ \texttt{longinteger} $2^{32}$). Запись из файла. Оценка числа шагов. Вывод в файл. Оценка числа шагов (14)}


\subsection{Представление длинного числа}

Всякое целое неотрицательное число $x$ может быть представлено в $m$-ичной системе счисления (при $m \geq 2)$ в виде $x=m^{k-1} x_{0}+m^{k-2} x_{1}+\ldots m x_{k-2}+x_{k-1}.$ При этом $k$ -- длина записи $m$-ичного представления числа $x, 0 \leq x_{i} \leq m-1$ при $i=0, \ldots, k-1$

\subsection{Запись из файла}

\textbf{Оценка числа шагов:} квадратичное от длины записи исходного числа в файле количество ''шагов''.

Пусть в файле записано десятичное число, заданное словом $a_{1} \cdots a_{n}\left(0 \leq a_{i} \leq 9\right)$. Требуется представить его динамическим массивом (или списком).

Под ''шагом'' понимается одна из следуюших операций:
считывание цифры из файла, запись цифры в целочисленный массив, выделение первой цифры многозначного числа и её удаление из него, приписывание цифры в конец числа. Заметим, что эти ''шаги'' не равнозначны, т.к. последние два требуют нахождения остатка от деления на $10$, а также умножения на 10 и сложения.

\subsection{Вывод в файл}

\textbf{Оценка числа шагов:} линейное от длины записи исходного числа количество ''шагов''.

При выводе числа необходимо помнить, что в каждом элементе массива, в котором хранится многоразрядное число, записана не последовательность цифр, а число, записанное этими цифрами. Поэтому число, десятичная запись которого меньше, чем длина записи выбранного нами основания $m$, необходимо дополнить ведущими нулями.

Под ''шагом'' будем понимать одну из следующих операций: запись ''макроцифры'' в символьную переменную, сравнение длины записи ''макроцифры'' с $\|m-1\|$, дополнение строки ведущим нулём.



\section{Сложение двух ''длинных'' положительных чисел. Оценка числа шагов (17)}

\textbf{Оценка числа шагов:} общее число ''шагов'' при сложении двух неотрицательных чисел не превосходит $3\max \{A[0], B[0]\}+1$, то есть составляет $\mathrm{O}(\max \{A[0], B[0]\})$

Чтобы сложить два неотрицательных многоразрядных числа, записанных в массивы $A$ и $B$, достаточно последовательно складывать по модулю $m$ числа, записанные в $A[i], B[i]$ и $d[i]$ для $i=$ $1, \ldots, \max \{A[0], B[0]\}$, где $d[1]=0$, при $i>1, d[i]-$ это $1$ (если $A[i-1]+B[i-1]+d[i-1]>m$) или 0 в противном случае.

При подсчёте числа шагов в этом разделе под ''шагом'' понимается одна из следующих операций: вычисление $A[i-1]+B[i-1]+d[i-1]$ $\bmod m$, проверка условия $A[i-1]+B[i-1]+d[i-1]>m$ и вычисление $d[i]$


\section{Предикаты равенства и неравенств ''длинных'' положительных чисел. Оценка числа шагов (18)}

\textbf{Оценка числа шагов:} если под ''шагом'' понимать количество сравнений ''макроцифр'', то обшее число ''шагов'' такой процедуры не превосходит $A[0]$. В общем случае число ''шагов'' вычисления каждого из четырёх предикатов не превосходит $\min \{A[0], B[0]\}$.


Оценим число шагов вычисления значений предикатов $x=y$ и $x<y$ для случая, когда $A[0]=B[0]$.

Начиная со старшего разряда (то есть с $A[A[0]]$ и $B[B[0]])$ сравниваем значения чисел в $A[i]$ и $B[i]$ до тех пор, пока они совпадают. Если для некоторого $i_{0} A\left[i_{0}\right] \neq B\left[i_{0}\right]$, то $x \neq y$. Если при этом $A\left[i_{0}\right]<B\left[i_{0}\right]$, то $x<y$, если $A\left[i_{0}\right]>B\left[i_{0}\right]$, то $x>y$.


\section{Вычитание двух ''длинных'' положительных чисел. Оценка числа шагов (18)}

\textbf{Оценка числа шагов:} общее число  ''шагов'' при вычитании двух положительных чисел не превосходит $4 \max \{A[0], B[0]\}+1$, то есть составляет $\mathrm{O}(\max \{A[0], B[0]\})$.

При подсчёте числа шагов в этом разделе под ''шагом'' понимается одна из следуюшци операций: вычисление $A[i-1]-B[i-1]-d[i-1]$ $(\bmod m)$, проверка условия $A[i-1]+B[i-1]+d[i-1]>0$ и вычисление $d[i]$. Кроме того, предварительно проверяется условие $x \geq y$.




\section{Умножение ''длинного'' числа на короткое. Оценка числа шагов (19)}

\textbf{Оценка числа шагов:} общее число операций не превосходит $\max \{1,2+6 A[0]+$ $\max \{1,3\}\}=5+6 A[0]$, то есть составляет $\mathrm{O}(A[0])$.

Здесь под шагом будем понимать одну из следуюших операций: умножение макроцифр, сложение макроцифр, вычисление неполного частного и остатка от деления результата предыдущих операций на $m$. B условном операторе после else выполняется одно присваивание и оператор цикла, в котором (помимо двух операций, необходимых для организации цикла) производятся: умножение, сложение, остатка от деления на $m$, вычисление неполного частного. Всего в операторе цикла 6 ''шагов''.



\section{Умножение ''длинных'' чисел. Оценка числа шагов (20)}

\textbf{Оценка числа шагов:} В предположении, что $B[0] \leq A[0]$ (это условие проверяется за 1 «шаг» и в противном случае можно умножать $B$ на $A$), получаем оценку $O(A[0] B[0])$.


\section{Деление ''длинных'' чисел. Оценка числа шагов (21)}

\textbf{Оценка числа шагов:} $O (A[0] B[0] \cdot(A[0]-B[0]))$

Будем подбирать неполное частное от деления чисел $x$ и $y$, записанных в массивах $A$ и $B$, делением промежутка, в котором оно может находиться, пополам. Пусть $L$ и $U$ -- нижняя и верхняя границы промежутка соответственно, $M=\left\lfloor\frac{L+U}{2}\right\rfloor$ -- целая часть середины промежутка, $z=y \cdot M-$ число, которое будем сравнивать с делимым.

При этом будем предполагать, что число, записанное в $A$, больше числа, затисанного в $B$ (в противном случае неполное частное равно 0, а остаток совпадает с делимым).







\section{Оценки числа шагов метода Гауса при действиях с ''длинными'' числами}


Итоговая асимптотика: $O(\min (n, m) \cdot n m)$

При $n=m$ эта оценка превращается в $O(n^{3})$
Для длинных чисел получается $O(M*n^{3})$. 

\section{Сортировки и оценки числа их шагов: Пузырёк. Сортировка вставками. Сортировка слияниями фон Неймана (25)}

\subsection{Пузырёк}

\textbf{Сложность:} $O(n^2)$

В теле циклов сравниваются значения $a[i]$ и $a[j]$. В случае необходимости содержание элементов массива меняются местами. Обмен значениями переменных $x$ и $y$ можно осуществить с помошью трёх операторов присваивания с использованием вспомогательной переменной $z: z:=x ; x:=y ; y:=z .^{3}$ Таким образом, в теле цикла каждый раз выполняется не более четырёх операций.


\subsection{Сортировка вставками}

\textbf{Сложность:} $O(n^2)$

Элементы входной последовательности просматриваются по одному, и каждый новый поступивший элемент размещается в подходящее место среди ранее упорядоченных элементов.

\subsection{Сортировка слияниями фон Неймана}

\textbf{Сложность:} $O(n \log_{2} n)$

Сортируемый массив разбивается на две части примерно одинакового размера; Каждая из получившихся частей сортируется отдельно, например — тем же самым алгоритмом; Два упорядоченных массива половинного размера соединяются в один.


\section{Алгоритмы на графах, различные способы представления графа в компьютере (28)}


\begin{itemize}
    \item $V$ -- произвольное конечное множество; 
    \item $E$ -- подмножество множества двуэлементных подмножеств множества $V$;
    \item $G=(V, E)$ -- граф с множеством вершин $V$ и множеством рёбер $E$;
    \item $A$ -- подмножество множества упорядоченных пар множества $V$; 
    \item $G=(V, A)$ -- орграф с множеством вершин $V$ и множеством дуг $A$;
    \item $n$ -- количество вершин в графе;
    \item $m$ -- количество рёбер в графе;
    \item $N(v)$ -- окружение вершины $v$, т.е. множество вершин, смежных c $v$;
    \item $O U T(v)$ -- множество вершин орграфа, непосредственно достижимых из $v$;
    \item $I N(v)$ -- множество вершин орграфа, из которых $v$ непосредственно достижима;
    \item $\operatorname{deg}(v)$ -- степень вершины $v$, т.е. количество вершин в окружении;
    \item $w_{i j}$ -- вес ребра $\left\{v_{i}, v_{j}\right\}$ или ребра $\left(v_{i}, v_{j}\right)$ во взвешеном графе.
\end{itemize}

\subsection{Матрица смежности}

Матрица смежности графа -- это квадратная матрица $A_{n \times n},$ элементы которой определены так:
$$
a_{i j}=\left\{\begin{array}{ll}
1 & \text { если }\left\{v_{i}, v_{j}\right\} \in E \\
0 & \text { иначе }
\end{array}\right.
$$


\subsection{Списки смежности}

Списки смежности -- это одномерный массив, $i$-ым элементом которого является список вершин, смежных с $v_{i}$, т.е. окружение вершшны $v_{i}$.



\subsection{Матрица инцидентности}

Матрица инцидентности графа -- это матрица $B_{n \times m}$, элементы которой определены так:
$$
b_{i j}=\left\{\begin{array}{ll}
1 & \text { если } v_{i} \in e_{j} \\
0 & \text { иначе }
\end{array}\right.
$$

\section{Алгоритм поиска в глубину. Оценки числа шагов в зависимости от способа представления графа (28)}

\begin{itemize}
    \item \textbf{Матрица смежности: } $O\left(n^{2}\right)$
При обходе графа в глубину или в ширину для каждой вершины необходимо проверить все (при обходе в глубину - постепенно, а при обходе в ширину - сразу) вершины, смежные с данной. При использовании матрицы смежности для одной вершины это можно сделать за $n$ проверок того, следует ли помешать вершины в стек или в очередь. Эта процедура выполняется для каждой из $n$ вершин. Этим объясняется то, что алгоритмы, основанные на обходе графа в глубину или в ширину при представлении графа матрицей смежности, имеют оценку числа шагов вида $O\left(n^{2}\right)$.

Для орграфа элементы матрицы смежности определяются так
$$
a_{i j}=\left\{\begin{array}{ll}
1 & \text { если }\left(v_{i}, v_{j}\right) \in A \\
0 & \text { иначе }
\end{array}\right.
$$
Рассуждениями, аналогичными таковым для не ориентированного графа, получаем оценку числа шагов вида $O\left(n^{2}\right)$.

    \item \textbf{Списки смежности: } $O(n+m)$. Для графов с разными свойствами эта оценка может быть видоизменена.

Если граф является дереном или лесом, то $m<n$ и оценка принимает вид $O(n)$.

Ecru rpaф полный, то $m=\frac{n(n-1)}{2}$ и оценка принимает вид $O\left(n^{2}\right)$. Если степени всех вершин графа не превосходят некоторой константы $C$ (сушественно меньшей, чем $n$), то $m \leq C n$ и оценка принимает вид $O(n)$.

Если граф связен и степени вершин произвольны, то $m \geq n-1$ и оценка принимает вид $O(m)$.

Для орграфа список смежности для вершины $v$ состоит из вершин, входяших в $O U T(v)$ (или в $I N(v))$, Поскольку $\sum_{v \in V}\|O U T(v)\|=\sum_{v \in V}\|I N(v)\|=m$, то рассуждениями, аналогичными для не ориентированного графа, получаем оценку числа шагов вида $O(n+m)$.
    \item \textbf{Матрица инцидентности: } $O(n m)$
\end{itemize}




\section{Алгоритм поиска в ширину. Оценки числа шагов в зависимости от способа представления графа (28)}

\begin{itemize}
    \item \textbf{Матрица смежности: } $O\left(n^{2}\right)$
        При обходе графа в глубину или в ширину для каждой вершины необходимо проверить все (при обходе в глубину - постепенно, а при обходе в ширину - сразу) вершины, смежные с данной. При использовании матрицы смежности для одной вершины это можно сделать за $n$ проверок того, следует ли помешать вершины в стек или в очередь. Эта процедура выполняется для каждой из $n$ вершин. Этим объясняется то, что алгоритмы, основанные на обходе графа в глубину или в ширину при представлении графа матрицей смежности, имеют оценку числа шагов вида $O\left(n^{2}\right)$.

Для орграфа элементы матрицы смежности определяются так
$$
a_{i j}=\left\{\begin{array}{ll}
1 & \text { если }\left(v_{i}, v_{j}\right) \in A \\
0 & \text { иначе }
\end{array}\right.
$$
Рассуждениями, аналогичными таковым для не ориентированного графа, получаем оценку числа шагов вида $O\left(n^{2}\right)$.

    \item \textbf{Списки смежности: } $O(n+m)$. Для графов с разными свойствами эта оценка может быть видоизменена.

Если граф является дереном или лесом, то $m<n$ и оценка принимает вид $O(n)$.

Ecru rpaф полный, то $m=\frac{n(n-1)}{2}$ и оценка принимает вид $O\left(n^{2}\right)$. Если степени всех вершин графа не превосходят некоторой константы $C$ (сушественно меньшей, чем $n$), то $m \leq C n$ и оценка принимает вид $O(n)$.

Если граф связен и степени вершин произвольны, то $m \geq n-1$ и оценка принимает вид $O(m)$.

Для орграфа список смежности для вершины $v$ состоит из вершин, входяших в $O U T(v)$ (или в $I N(v))$, Поскольку $\sum_{v \in V}\|O U T(v)\|=\sum_{v \in V}\|I N(v)\|=m$, то рассуждениями, аналогичными для не ориентированного графа, получаем оценку числа шагов вида $O(n+m)$.
    \item \textbf{Матрица инцидентности: } $O(n m)$
\end{itemize}




\section{Задачи, решаемые с помощью этих алгоритмов: выделение компонент связности; проверка на двудольность и выделение долей; выделение остова графа}

\subsection{Выделение компонент связности}

Компонента связности графа $G$ -- максимальный (по включению) связный подграф графа $G$. Другими словами, это подграф $G(U)$, порождённый множеством $U \subseteq V(G)$ вершин, в котором для любой пары вершин $u, v \in U$ в графе $G$ существует $(u, v)$-цепь и для любой пары вершин $u \in U, w \notin U$ не существует $(u, w)$-цепи.

Для выделения компонент связности можно использовать поиск в ширину или поиск в глубину. При этом затраченное время будет \textbf{линейным} от суммы числа вершин и числа рёбер графа.

\subsection{Проверка на двудольность и выделение долей}

\textbf{Двудольный граф} или биграф в теории графов -- это граф, множество вершин которого можно разбить на две части таким образом, что каждое ребро графа соединяет какую-то вершину из одной части с какой-то вершиной другой части, то есть не существует рёбер между вершинами одной и той же части.

Для того, чтобы проверить граф на предмет двудольности, достаточно в каждой компоненте связности выбрать любую вершину и помечать оставшиеся вершины во время обхода графа (например, поиском в ширину) поочерёдно как чётные и нечётные. Если при этом не возникнет конфликта, все чётные вершины образуют множество $U$, а все нечётные -- $V$.

\subsection{Выделение остова графа}

\textbf{Остовное дерево графа} -- это дерево, подтраф данного графа, с тем же числом вершин, что и у исходного графа. Неформально говоря, остовное дерево получается из исходного графа удалением максимального числа рёбее, входящих в циклы, но без нарушения связности графа. остовное дерево включает в себя все $n$ вершин исходного графа и содержит $n-1$ ребро.

Остовное дерево может быть построено практически любым алгоритмом обхода графа, например поиском в глубину или поиском в ширину. Оно состоит из всех пар рёбер $(u, v)$, таких, что алгоритм, просматривая вершину $u$, обнаруживает в её списке смежности новую, не обнаруженную ранее вершину $v$.



\section{Нахождение остова минимального веса. Метод Р. Прима. Оценки числа шагов (32)}

Оценки числа шагов работы этого алгоритма аналогичны оценкам числа шагов алгоритма Дейкстры за исключением того, что в п. 4 не выполняется операция сложения. В результате получаем
$$ O\left(n^{2}+m\right)=O\left(n^{2}\right) $$



\section{Алгоритм Дейкстры поиска кратчайшего пути. Оценки числа шагов (31)}

\textbf{Сложность: } $O(n^2 + m) = O(n^2)$

Задан взвешенный орграф $G=(V, A)$ (все веса $w_{i j}$ положительны) и две выделенные вершины $s$ -- старт и $f$ -- финиш. Требуется найти кратчайший путь из $s$ в $f$.


\section{Нахождение циклов и мостов в графе. Оценки числа шагов}

\subsection{Циклы}

\textbf{Замкнутый обход} состоит из последовательности вершин, начинающейся и заканчивающейся в той же самой вершине, и каждые две последовательные вершины в последовательности смежны.

\textbf{Сложность: } $O(n + m)$

Неориентированный граф имеет цикл в том и только в том случае, когда поиск в глубину (DFS) находит ребро, которое приводит к уже посещённой вершине (обратная дуга). Таким же образом, все обратные рёбра, которые алгоритм DFS обнаруживает, являются частями циклов. Для неориентированных графов требуется только время $O(n)$ для нахождения цикла в графе с $n$ вершинами, поскольку максимум $n-1$ рёбер могут быть рёбрами дерева.

\subsection{Мосты}

\textbf{Мост} -- ребро в теории графов, удаление которого увеличивает число компонент связности. Эквивалентное определение -- ребро является \textbf{мостом} в том и только в том случае, если оно не содержится ни в одном цикле.

\textbf{Сложность: } $O(n + m)$

\section{Эйлеров цикл.  Оценки числа шагов}

\textbf{Эйлеров путь} -- это путь в графе, проходящий через все его рёбра.

\textbf{Эйлеров цикл} -- это эйлеров путь, являющийся циклом.

Граф называется эйлеровым, если он содержит эйлеров цикл.
Мультиграф - граф, в котором разрешается присутствие кратных рёбер.
Асимптотика задачи поиска эйлерова пути $\mathrm{O}(\mathrm{m})$ при использовании списков смежности.


\section{Гамильтонов цикл.  Оценки числа шагов}

\textbf{Гамильтонов цикл} -- цикл , который проходит через каждую вершину данного графа ровно по одному разу.

\textbf{Гамильтонов граф} -- граф, содержащий гамильтонов цикл.

Асимптотика задачи нахождения гамильтонова цикла в графе $\mathrm{O}\left(\mathrm{d}^{\wedge}(\mathrm{n}-1)\right)$. Где $\mathrm{d}$ это максимальная степень вершины в графе - $1 .$

\section{Алгоритм генерации всех независимых множеств.  Оценки числа шагов (не будут спрашивать)}

\section{Теорема о НМ, ВП, КЛИКА.  Оценки числа шагов}

Алгоритм для поиска клик подойдет для поиска НМ. Ищем клики в $G$ (дополнение), в $G$ это будут независимые множества Алгоритм Брона-Кербоша -- метод ветвей и границ для поиска всех клик. Вычислительная сложность алгоритма линейна относительно количества клик в графе. В худшем случае алгоритм работает за $О(3^{n/3})$ шагов.

    \begin{figure}[h!]
        \includegraphics[width=1\textwidth]{./images/21.png}
        \centering
    \end{figure}

\section{Отличия между интуитивным и математическим понятиями}
\section{Машины Тьюринга и их модификации. Тезис Тьюринга-Чёрча}
\section{Теорема о числе шагов МТ, моделирующей работу k-ленточной МТ}
\section{Недетерминированные МТ.  Теорема о числе шагов МТ, моделирующей работу недетерминированной МТ}
\section{Понятия сложности алгоритма от данных, сложность алгоритма, сложность задачи. Верхняя и нижняя оценки сложности}
\section{Соотношение между временем работы алгоритма требуемой памятью}
\section{Классы алгоритмов и задач. Схема обозначений}
\section{Классы $P$, $NP$ и $P-SPACE$. Соотношения между этими классами}
\section{Полиномиальная сводимость и полиномиальная эквивалентность}
\section{Полиномиальная сводимость задачи ГЦ к задаче КОМИВОЯЖЁР}
\section{Классы эквивалентности по отношению полиномиальной эквивалентности. Класс P – пример такого класса}
\section{NP-полные задачи. Класс NP-полных задач — класс эквивалентности по отношению полиномиальной эквивалентности}
\section{Задача ВЫПОЛНИМОСТЬ (ВЫП).Теорема Кука}
\section{Задача 3-ВЫПОЛНИМОСТЬ (3-ВЫП). Её NP-полнота}
\section{Задачи ВЕРШИННОЕ ПОКРЫТИЕ (ВП), НЕЗАВИСИМОЕ МНОЖЕСТВО (НМ), КЛИКА.  NP-полнота задачи ВП.  Полиномиальная эквивалентность этих трёх задач}
\section{NP-полнота задач ГЦ и ГП (без доказательства)}
\section{NP-полнота задач 3-С и РАЗБИЕНИЕ (без доказательства)}
\section{Метод сужения доказательства NP-полноты}
\section{''Похожие'' задачи и их сложность}
\section{Анализ подзадач}
\section{Алгоритм решения задачи РАЗБИЕНИЕ}
\section{Задачи с числовыми параметрами. Псевдополиномиальные задачи}

\end{document}
